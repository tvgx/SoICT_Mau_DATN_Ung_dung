\documentclass[../main.tex]{subfiles}
\begin{document}

Sau khi đã trình bày các công nghệ sử dụng trong Chương 3, chương này sẽ mô tả chi tiết quá trình thiết kế và triển khai hệ thống thực tế. Nội dung bao gồm thiết kế kiến trúc với các layer cụ thể, thiết kế cơ sở dữ liệu với các collection trong MongoDB, thiết kế và triển khai API endpoints, quy trình triển khai hệ thống thực tế, và đánh giá kết quả đạt được.

\section{Thiết kế kiến trúc hệ thống}
\label{section:4.1}

Hệ thống được thiết kế theo kiến trúc phân tầng rõ ràng, áp dụng mô hình MVC cho Backend và component-based architecture cho Frontend.

Về kiến trúc Backend, hệ thống được tổ chức theo mô hình MVC mở rộng với năm tầng chính. Tầng Routes định nghĩa các endpoint API và HTTP methods. File index.js trong thư mục routes tập trung tất cả các route modules bao gồm auth, products, categories, orders, users, wishlist, events, dashboard, payment và upload. Mỗi route module được mount vào một prefix riêng như /api/auth, /api/products. Một số routes như authentication và products là public, trong khi orders, wishlist và dashboard yêu cầu authentication.

Tầng Controllers nhận request từ routes và điều phối xử lý. Các controller files được tổ chức theo chức năng như auth.controller.js, product.controller.js, order.controller.js, payment.controller.js, dashboard.controller.js. Controllers sử dụng catchAsync wrapper để xử lý errors một cách đồng bộ. Controllers validate request data, gọi services để xử lý logic, và format response trả về client.

Tầng Services chứa business logic của hệ thống. Services như product.service.js xử lý logic tìm kiếm, lọc sản phẩm, order.service.js quản lý đơn hàng và tính toán tổng tiền, dashboard.service.js tính toán số liệu thống kê, user.service.js quản lý người dùng, token.service.js tạo và verify JWT tokens. Services tương tác với Models để thao tác database.

Tầng Models định nghĩa data schema với Mongoose. User model với các trường name, email, password hash, role, và timestamps. Product model bao gồm name, description, price, oldPrice, stock, category reference, imageUrl, images array, rating, reviewCount, và tags. Order model có userId reference, items array chứa embedded product info, totalPrice được tính tự động qua middleware, status với các giá trị cart, pending, paid, shipped, delivered, cancelled, shippingAddress object và paymentResult. Category model đơn giản với name và description. Event model và Wishlist model cho các chức năng khuyến mãi và danh sách yêu thích.

Tầng Middlewares xử lý các tác vụ chung. Auth middleware kiểm tra JWT token trong header, verify token với secret key, extract user info từ payload, và attach vào request object. Upload middleware sử dụng multer để xử lý file upload với validation về file size và type.

Về cấu hình hệ thống, database connection được quản lý trong db.js với mongoose.connect, passport configuration trong passport.js thiết lập local strategy và JWT strategy, payos.config.js khởi tạo PayOS client với credentials từ environment variables.

Về kiến trúc Frontend, hệ thống được xây dựng với Next.js App Router. Tầng App Router trong thư mục src/app định nghĩa pages và routing structure. Tầng Components chứa UI components được tổ chức theo atomic design, reusable components như Button, Input được sử dụng nhiều nơi. Tầng Store sử dụng Zustand để quản lý state global. cart.store.ts quản lý giỏ hàng với items array, selectedItemIds cho checkout, các actions như addItem, removeItem, updateQuantity, và persist vào localStorage. auth.store.ts quản lý user authentication state và token. wishlist.store.ts quản lý danh sách yêu thích. Tầng Lib chứa utilities như api-client.ts tạo axios instance với base URL và interceptors.

Về luồng xử lý request, khi user thực hiện action trên UI, component gọi action từ Zustand store hoặc trực tiếp call API, store action gọi API client với endpoint tương ứng, API client gửi HTTP request đến Backend kèm JWT token, Backend routes nhận request và forward đến controller, controller validate data và gọi service, service xử lý logic và tương tác model, model thao tác MongoDB database, kết quả được trả ngược qua service, controller, response về Frontend, store cập nhật state với data mới, component re-render với state updated.

Kiến trúc này đảm bảo separation of concerns rõ ràng, dễ test từng layer độc lập, dễ maintain và extend chức năng mới, và scale tốt khi hệ thống lớn lên.

\section{Thiết kế cơ sở dữ liệu}
\label{section:4.2}

Cơ sở dữ liệu MongoDB được thiết kế với sáu collections chính, mỗi collection được định nghĩa bằng Mongoose schema với validation rules và indexes.

Collection users lưu trữ thông tin người dùng và quản trị viên. Schema bao gồm trường name kiểu String bắt buộc để lưu tên người dùng, email kiểu String bắt buộc unique và lowercase để đảm bảo không trùng lặp, password kiểu String bắt buộc với select false để mặc định không trả về khi query, role kiểu String enum với giá trị user hoặc admin mặc định là user, và timestamps tự động thêm createdAt và updatedAt. Schema có pre-save middleware để hash password bằng bcrypt với salt rounds 10 trước khi lưu, và method comparePassword để so sánh password khi login.

Collection products lưu thông tin sản phẩm với schema phức tạp. Các trường bao gồm name kiểu String required và trim, description kiểu String required, price kiểu Number required với validation min 0, oldPrice kiểu Number cho giá gốc khi sale, stock kiểu Number required với min 0 mặc định 0, category kiểu ObjectId reference đến Category collection với index, imageUrl kiểu String required cho ảnh chính, images array String cho gallery, rating kiểu Number từ 0 đến 5 mặc định 0, reviewCount kiểu Number mặc định 0, và tags array String với index để lọc như flash-sale hoặc best-selling. Schema có text index trên name và description để hỗ trợ full-text search.

Collection categories quản lý danh mục sản phẩm với schema đơn giản. Trường name kiểu String required unique và trim, description kiểu String trim, imageUrl kiểu String, và timestamps.

Collection orders quản lý đơn hàng với schema phức tạp nhất. Trường userId kiểu ObjectId reference đến User với index, items là array của OrderItemSchema embedded chứa productId reference, name, price, imageUrl được copy từ Product, và quantity min 1, totalPrice kiểu Number required mặc định 0 được tính tự động, status kiểu String enum với các giá trị cart cho giỏ hàng đang hoạt động, pending cho đơn hàng chờ thanh toán, paid cho đã thanh toán, shipped cho đang giao hàng, delivered cho đã giao, và cancelled cho đã hủy, shippingAddress object chứa street, city, postalCode, country, và paymentResult object với id và status từ PayOS. Schema có pre-save middleware tự động tính totalPrice bằng cách sum price nhân quantity của tất cả items.

Collection events lưu các sự kiện khuyến mãi và Collection wishlist quản lý danh sách yêu thích của người dùng với reference đến userId và productId array.

Về quan hệ giữa các collections, quan hệ One-to-Many giữa Category và Products được implement bằng ObjectId reference, một category có nhiều products nhưng mỗi product chỉ thuộc một category. Quan hệ One-to-Many giữa User và Orders, một user có nhiều orders, mỗi order thuộc một user. Quan hệ Many-to-Many giữa Product và Order được implement qua embedded OrderItemSchema, product info được denormalize vào order items để tránh phải join khi hiển thị đơn hàng. Quan hệ One-to-One giữa User và Wishlist, mỗi user có một wishlist riêng.

Về indexing strategy, index được tạo cho các trường thường xuyên query. Email trong users có unique index, category trong products có index để lọc theo danh mục, tags trong products có index để lọc flash sale hoặc best selling, text index trên name và description của products để full-text search, userId trong orders có index để lấy đơn hàng của user nhanh, và status trong orders có index để filter theo trạng thái.

Thiết kế này tối ưu cho use cases của e-commerce như tìm kiếm sản phẩm nhanh, lọc theo category và tags hiệu quả, lấy orders của user không cần scan toàn bộ collection, và embedded items trong orders giảm số lượng queries khi hiển thị chi tiết đơn hàng.

\section{Thiết kế và triển khai API}
\label{section:4.3}

Hệ thống cung cấp RESTful API với các endpoint được nhóm theo chức năng rõ ràng. Tất cả API đều có prefix /api.

Nhóm Authentication API xử lý đăng ký và đăng nhập. POST /api/auth/register nhận request body với name, email, password, validate dữ liệu đầu vào, kiểm tra email đã tồn tại chưa, hash password với bcrypt, tạo user record mới, tạo JWT token, và trả về user info và token. POST /api/auth/login nhận email và password, tìm user trong database kèm password field, so sánh password với bcrypt comparePassword method, tạo JWT token nếu match, và trả về user info và token. GET /api/auth/me yêu cầu authentication với JWT trong header, middleware verify token và attach user vào request, và trả về user info hiện tại.

Nhóm Product API quản lý sản phẩm. GET /api/products là public route để lấy danh sách sản phẩm, hỗ trợ query params như page và limit cho pagination, category cho lọc theo danh mục, minPrice và maxPrice cho lọc theo giá, search cho full-text search, sort cho sắp xếp theo price, name, createdAt, và trả về products array với pagination metadata. GET /api/products/:id lấy chi tiết một sản phẩm với populate category info. POST /api/products yêu cầu admin role, nhận name, description, price, stock, category, imageUrl, validate dữ liệu, tạo product mới, và trả về product created. PUT /api/products/:id yêu cầu admin role để update product. DELETE /api/products/:id yêu cầu admin role để xóa sản phẩm.

Nhóm Category API quản lý danh mục. GET /api/categories là public để lấy tất cả danh mục. POST /api/categories yêu cầu admin role nhận name và description để tạo category mới. PUT /api/categories/:id và DELETE /api/categories/:id yêu cầu admin role.

Nhóm Order API quản lý đơn hàng và giỏ hàng. GET /api/orders/cart yêu cầu authentication, tìm order với userId và status cart, populate product info, và trả về cart items. POST /api/orders/cart thêm sản phẩm vào giỏ nhận productId và quantity, tìm existing cart order, nếu chưa có thì tạo mới với status cart, nếu product đã có trong items thì tăng quantity, nếu chưa có thì push vào items array, và save order. DELETE /api/orders/cart/:productId xóa item khỏi giỏ. PUT /api/orders/cart/:productId cập nhật quantity. POST /api/orders/checkout chuyển cart thành pending order nhận shippingAddress, tìm cart order, validate items và stock, cập nhật status thành pending, và trả về order info. GET /api/orders lấy danh sách orders của user với filter theo status. GET /api/orders/:id lấy chi tiết order.

Nhóm Payment API tích hợp PayOS. POST /api/payment/create-link nhận orderId, amount, description, returnUrl, cancelUrl, validate required fields, tạo paymentData object với orderCode numeric, amount numeric, description, returnUrl mặc định FRONTEND\_URL/payment/success, và cancelUrl mặc định FRONTEND\_URL/payment/failed, gọi payOS.createPaymentLink với paymentData, và trả về checkoutUrl và paymentLinkId. POST /api/payment/webhook nhận webhook từ PayOS, log webhook data, verify signature nếu cần, kiểm tra code 00 là success, tìm order theo orderCode, cập nhật paymentStatus và status, save order, và trả về success response. GET /api/payment/:paymentLinkId lấy thông tin payment từ PayOS.

Nhóm Dashboard API cho admin. GET /api/dashboard/stats yêu cầu admin role, tính tổng doanh thu từ orders paid, đếm số orders theo status, tìm top selling products, tính revenue theo ngày trong 7 ngày gần nhất, và trả về statistics object.

Nhóm Wishlist API quản lý danh sách yêu thích. GET /api/wishlist lấy wishlist của user. POST /api/wishlist thêm product vào wishlist. DELETE /api/wishlist/:productId xóa khỏi wishlist.

Nhóm Event API quản lý sự kiện khuyến mãi. GET /api/events lấy danh sách events. POST, PUT, DELETE yêu cầu admin role.

Nhóm Upload API xử lý upload hình ảnh. POST /api/upload sử dụng multer middleware, validate file type là image, lưu file vào thư mục public/uploads, và trả về file URL.

Về authentication và authorization, hệ thống sử dụng JWT token trong Authorization header với format Bearer token. Middleware auth.middleware.js extract token từ header, verify token với JWT\_SECRET, decode để lấy userId và role, tìm user trong database, attach user object vào req.user, và next nếu valid hoặc return 401 Unauthorized. Admin endpoints kiểm tra thêm req.user.role === admin.

Về error handling, tất cả controllers wrap bằng catchAsync utility, errors được catch và forward đến error middleware, ApiError class chuẩn hóa error response với statusCode và message, và error middleware format response JSON với code, message, và stack trace trong development.

API design này tuân thủ RESTful principles, sử dụng HTTP methods đúng nghĩa với GET cho read, POST cho create, PUT cho update, DELETE cho delete, status codes chuẩn với 200 OK, 201 Created, 400 Bad Request, 401 Unauthorized, 404 Not Found, 500 Internal Server Error, và response format nhất quán với JSON.

\section{CI/CD Pipeline - Thiết kế và triển khai}
\label{section:4.4}

Một trong những thành phần quan trọng nhất của dự án là triển khai quy trình CI/CD tự động hóa, giúp giảm thời gian deployment từ 30 phút xuống còn 5 phút và đảm bảo chất lượng code thông qua automated testing.

\subsection{Kiến trúc CI/CD tổng thể}
\label{subsection:4.4.1}

Hệ thống CI/CD được thiết kế theo mô hình pipeline tự động với flow từ code commit đến production deployment mà không cần intervention thủ công.

Về overall architecture, khi developer push code lên GitHub, webhook tự động trigger GitHub Actions workflows. Pipeline chạy các giai đoạn tuần tự: code quality checks với ESLint và Prettier, TypeScript type checking cho frontend, unit tests và integration tests, build Docker image cho backend hoặc Next.js build cho frontend. Nếu tất cả checks pass, code được deploy tự động đến cloud platforms với Render cho backend và Vercel cho frontend. Health checks được thực hiện sau deployment để verify application hoạt động, và notifications được gửi về deployment status.

Về branching strategy, main branch là production branch với auto-deploy enabled và protected branch requiring PR reviews. Develop branch dùng cho staging environment với tự động deploy để test integration. Feature branches được tạo từ develop, require PR to merge, và trigger preview deployments trên Vercel cho frontend.

Deployment flow chi tiết như sau. Developer tạo feature branch từ develop, thực hiện changes và commit code locally, push to GitHub và tạo pull request. GitHub Actions tự động trigger với các jobs lint, test, và build verification. Nếu checks pass, code được review và merge vào develop, triggering staging deployment. Sau khi test trên staging, develop được merge vào main, triggering production deployment với full test suite, Docker build và push, automated deployment to Render và Vercel, health check verification, và notification về deployment status.

\subsection{Docker containerization}
\label{subsection:4.4.2}

Backend được containerize với Docker để đảm bảo tính nhất quán giữa các môi trường và dễ dàng deploy lên cloud platforms.

Về multi-stage Dockerfile implementation, Stage 1 là builder stage với base image node:18-alpine được chọn vì lightweight chỉ 40MB so với node:18 standard 900MB. Working directory được set /app. Package files package.json và package-lock.json được copy trước để leverage Docker layer caching. Dependencies được install với npm ci --only=production để reproducible builds và loại bỏ devDependencies. Source code được copy sau để tránh rebuild dependencies khi code thay đổi.

Stage 2 là production stage với base image node:18-alpine tương tự. NODE\_ENV được set production để optimize runtime. Non-root user nodejs được tạo với addgroup và adduser để enhance security, tránh chạy application với root privileges. Chỉ production artifacts từ builder stage được copy với ownership nodejs:nodejs. User được switch sang nodejs trước khi start app. Port 8000 được expose cho application. Health check được define với interval 30 seconds, timeout 10 seconds, start period 40 seconds để container initialize, và retries 3 lần trước khi mark unhealthy. Health check command chạy node script call /api/health endpoint. CMD được set node server.js để start application.

Về optimization techniques applied, multi-stage build giảm final image size từ 1.2GB xuống 280MB, giảm 77 phần trăm. Docker ignore file loại bỏ node\_modules, tests, documentation, và development files khỏi build context, giảm build time từ 45 seconds xuống 22 seconds. Layer ordering được optimize với frequently changing code ở cuối để maximize cache hits. npm ci được dùng thay vì npm install để faster và deterministic installs. Alpine Linux base image được chọn thay vì standard Debian base để minimal size.

Security considerations bao gồm non-root user execution để prevent privilege escalation, minimal base image reducing attack surface, no secrets in image với environment variables injected at runtime, health check để automatic container restart nếu unhealthy, và read-only filesystem where possible.

Build performance metrics cho thấy initial build 45 seconds, với cache 22 seconds improvement 51 percent, incremental builds chỉ 8-12 seconds khi code changes, và image pull time 15 seconds với compressed layers.

\subsection{GitHub Actions workflows}
\label{subsection:4.4.3}

Automated CI/CD được implement với GitHub Actions workflows để orchestrate toàn bộ pipeline từ code commit đến deployment.

Về backend workflow trong file .github/workflows/backend-deploy.yml, triggers được config on push to main or develop branches với paths filter backend/** để chỉ run khi backend code thay đổi, và on pull\_request to main or develop để validate trước khi merge.

Job 1 là lint check chạy trên ubuntu-latest với steps checkout code, setup Node.js 18.x với npm cache enabled, install dependencies với npm ci, và run ESLint với npm run lint. Job này fail nếu có linting errors, ensuring code quality standards.

Job 2 là test với strategy matrix testing trên Node 18.x và 20.x để ensure compatibility. Steps tương tự lint nhưng thêm run unit tests với npm test và run integration tests nếu có. Coverage reports được upload as artifacts với retention 7 days. Job này requires lint pass trước khi run.

Job 3 là build Docker image chỉ run nếu push to main hoặc develop. Setup Docker Buildx được dùng cho advanced build features. Docker build với cache from GitHub Actions cache để speed up builds. Built image được test bằng cách run container, wait 10 seconds, check logs, và verify health endpoint responds. Container được stop sau test.

Job 4 là deploy to production chỉ run nếu ref is main branch và requires build job pass. Render deployment trigger automatically vì connected to GitHub repository. Script wait 30 seconds for deployment, run health check curl https://pr3-backend.onrender.com/api/health, và fail nếu health check không return 200. Notification step chạy always để report deployment status.

Job 5 là deploy to staging cho develop branch với similar flow nhưng target staging environment.

Về frontend workflow trong frontend-deploy.yml, triggers tương tự với paths filter frontend/**. Job 1 lint and type check với ESLint và TypeScript compiler tsc --noEmit để catch type errors trước build. Job 2 build Next.js application với npm run build, environment variable NEXT\_PUBLIC\_API\_GATEWAY\_URL được inject, và build artifacts uploaded for potential reuse. Job 3 run component tests với React Testing Library hoặc Jest. Job 4 deploy to Vercel production tự động cho main branch vì Vercel connected to repository. Job 5 deploy preview cho pull requests và develop branch, với preview URLs automatically commented on PRs.

Về workflow optimizations, dependency caching với actions/setup-node cache npm giảm install time từ 60s xuống 10s. Matrix builds cho phép parallel testing across Node versions. Conditional job execution với if statements prevent unnecessary runs. Build artifact upload và reuse across jobs. GitHub Actions cache cho Docker layers và npm packages. Path filters ensure workflows chỉ run khi relevant files change.

Về secrets management, GitHub Secrets store sensitive values như VERCEL\_ORG\_ID, VERCEL\_PROJECT\_ID, NEXT\_PUBLIC\_API\_GATEWAY\_URL. Secrets không appear trong logs hoặc artifacts. Principle of least privilege với separate secrets cho staging và production. Regular secret rotation quarterly. No hardcoded secrets trong workflow files.

\subsection{Environment configuration}
\label{subsection:4.4.4}

Application deployment yêu cầu quản lý environment variables cẩn thận giữa các môi trường development, staging và production.

Về environment separation strategy, Development environment chạy locally với .env.development file, MongoDB local instance hoặc Atlas dev cluster, reduced logging, và debug mode enabled. Staging environment trên develop branch với Render staging service, MongoDB Atlas staging cluster, moderate logging, và similar to production config. Production environment trên main branch với Render production service, MongoDB Atlas production cluster với replication, comprehensive error logging, và optimized performance settings.

Backend environment variables classification gồm core variables PORT default 8000, NODE\_ENV development hoặc production, FRONTEND\_URL for CORS configuration. Database MONGO\_URI connection string to MongoDB Atlas. Authentication JWT\_SECRET for signing tokens, JWT\_EXPIRATION default 7 days. Payment integration PAYOS\_CLIENT\_ID, PAYOS\_API\_KEY, PAYOS\_CHECKSUM\_KEY from PayOS dashboard. Logging LOG\_LEVEL info for production, debug for development.

Frontend environment variables include API configuration NEXT\_PUBLIC\_API\_GATEWAY\_URL pointing to backend API, prefixed NEXT\_PUBLIC for client-side access. Build optimization NEXT\_PUBLIC\_ENV production hoặc staging. Analytics NEXT\_PUBLIC\_ANALYTICS\_ID if using Google Analytics.

Về configuration management best practices, .env.example file provides template với placeholder values. .env files are gitignored để prevent committing secrets. Platform-specific configuration on Render dashboard for backend và Vercel dashboard for frontend. Environment validation on application startup checking required variables exist, validating format and values, và fail fast if misconfigured. Type-safe env with TypeScript typed environment variables và compile-time checks.

Configuration loading sử dụng dotenv package load .env files, process.env access variables, default values with fallbacks, và validation schema with Joi hoặc Zod.

\subsection{Deployment automation}
\label{subsection:4.4.5}

Automated deployment lên các cloud platforms giảm thời gian deployment và loại bỏ human errors.

Về Render deployment for backend, initial setup connect GitHub repository select backend folder, set build command docker build or npm install, set start command node server.js, configure environment variables, và enable auto-deploy on main branch. Auto-deploy mechanism monitors main branch for changes, triggers build on new commits, pulls latest code, runs build command hoặc builds Docker image, starts new instance, performs health checks at /api/health, switches traffic to new instance nếu healthy, và terminates old instance. Zero-downtime deployment achieved through rolling deployment strategy với new instance starts first, health check passes before traffic switch, gradual traffic migration, và old instance remains until new is verified.

Render configuration includes health check path /api/health với HTTP GET method, expected status 200, timeout 10 seconds, interval 30 seconds. Auto-scaling enabled on paid plans. Environment variables injected securely. Build logs available in dashboard. Deployment history with rollback capability.

Về Vercel deployment for frontend, setup process import GitHub project, framework auto-detected as Next.js, build settings auto-configured, environment variables added, và connect domain nếu có. Auto-deploy features production deployment on main branch push, preview deployments cho mỗi PR với unique URLs, instant rollbacks to previous deployments, edge network distribution globally, và automatic HTTPS certificates.

Vercel optimizations include automatic code splitting, image optimization with next/image, static file caching with long TTL, incremental static regeneration, và serverless functions for API routes.

Deployment verification process includes health check endpoints /api/health for backend return OK status, application accessible at domain, no errors in logs, metrics normal trong dashboard, và smoke tests cho critical flows.

Về deployment metrics tracking, deployment frequency average 2-3 per day, lead time from commit to production less than 15 minutes, deployment success rate 97 percent, mean time to recovery less than 5 minutes, và rollback time less than 2 minutes.

Error handling and recovery includes failed builds stop deployment, failed health checks trigger rollback, error notifications via email hoặc Slack, automatic retries for transient failures, và manual intervention for persistent issues.

\subsection{Monitoring và health checks}
\label{subsection:4.4.6}

Production monitoring đảm bảo application reliability và quick issue detection.

Health check implementation trong backend tại endpoint GET /api/health return JSON với status ok, timestamp ISO string, uptime in seconds, và database connectivity status. Implementation code check MongoDB connection is ready, check critical services available, respond within timeout, và return appropriate status codes.

Về monitoring strategies, application health Render performs automatic health checks every 30 seconds, restarts container if unhealthy after 3 failed checks, alerts on repeated failures, và tracks uptime metrics. Application metrics tracked include request count và response times, error rates by endpoint, database query performance, memory và CPU usage, và active connections.

Logging implementation với structured logging using Winston với levels error, warn, info, debug. Log format JSON in production cho easy parsing, human-readable in development. Log rotation daily với max 14 days retention. Error tracking captures stack traces, request context, user information nếu authenticated, và environment details.

Frontend monitoring với Vercel Analytics track page views và navigation, build times và deployment frequency, edge network performance, và function invocations. Browser error tracking captures client-side errors, API request failures, và performance metrics.

Về alerting and notifications, critical alerts for application down, database connection lost, high error rate spike, deployment failures. Warning alerts for elevated response times, high memory usage, unusual traffic patterns. Notification channels email for critical issues, dashboard for all events, và Slack integration optional.

Performance monitoring tracks API response time p50 less than 200ms, p95 less than 500ms, p99 less than 1 second. Database query time p95 less than 100ms. Frontend metrics First Contentful Paint less than 1.5s, Time to Interactive less than 3s, Cumulative Layout Shift less than 0.1.

Log aggregation strategy consolidate logs from Render, Vercel, database tại centralized location. Search và filter capabilities. Real-time log streaming for debugging. Archive older logs for compliance.

\subsection{Rollback và recovery procedures}
\label{subsection:4.4.7}

Khả năng rollback nhanh chóng là critical cho production reliability.

Rollback strategies gồm ba approaches. Git revert thực hiện git revert commit-hash, git push origin main, automated deployment triggers new build with reverted code. Platform rollback cho Render access dashboard select service click previous deployment và promote to current, hoặc Vercel select previous deployment và click promote to production, instant rollback without rebuild. Manual intervention trong extreme cases deploy known good version manually, bypass CI/CD if necessary, investigate issue after service restored.

Về rollback decision criteria, triggers include critical bugs affecting users, security vulnerabilities discovered, performance degradation severe, data corruption detected, và third-party service integration broken. Decision process assess impact severity, determine if quick fix possible, decide rollback or forward fix, execute rollback if severity warrants.

Recovery procedures follow steps identify issue through monitoring alerts or user reports, assess severity và impact scope, decide rollback or patch forward, execute rollback if needed via appropriate method, verify service restored with health checks và smoke tests, communicate status to stakeholders, investigate root cause post-incident, implement fix và deploy again with proper testing.

Testing rollback procedures practice rollback quarterly, document procedures clearly, train team on rollback process, verify rollback speed meets targets.

Database migration rollback considerations include migrations are potentially irreversible, backup before migration critical, test migration on staging first, implement forward migrations carefully, have rollback migration ready, monitor database metrics post-migration.

Blue-green deployment alternative strategy maintain two identical environments blue currently serving production và green new version deployed here, test green environment thoroughly, switch traffic from blue to green, keep blue available for instant rollback, decommission blue after green proven stable.

\section{Development và deployment workflow}
\label{section:4.5}

Quy trình development và deployment được tối ưu hóa để hỗ trợ team productivity và code quality.

Về local development setup, Backend requires Node.js 18 or higher, MongoDB local hoặc Atlas, package.json dependencies installed, .env file configured từ .env.example template, và run npm run dev:backend để start với nodemon. Frontend requires Node.js 18 or higher, backend running và accessible, package.json dependencies installed, .env.local configured với NEXT\_PUBLIC\_API\_GATEWAY\_URL, và run npm run dev:frontend để start Next.js dev server với HMR.

Git workflow best practices include branch naming feature/feature-name for new features, bugfix/issue-description for bug fixes, hotfix/critical-fix for production hotfixes. Commit messages follow conventional commits format feat: add new feature, fix: resolve bug, docs: update documentation, style: formatting changes, refactor: code restructuring, test: add tests, chore: maintenance tasks. Pull request process create PR from feature to develop, add description và context, request review from team, address review comments, merge after approval và passing checks.

Code review checklist verify functionality works as intended, code follows style guidelines, tests included và passing, no security vulnerabilities, performance acceptable, documentation updated, no merge conflicts.

Testing strategy local development run npm test for unit tests, npm run test:integration for API tests, manual testing in browser or Postman. CI pipeline automated tests run on every push, required to pass before merge, coverage reports generated. Pre-production testing on staging environment, full user flows tested, integration with external services verified.

\section{Kiểm thử và đánh giá}
\label{section:4.6}

Hệ thống được kiểm thử qua nhiều cấp độ để đảm bảo chất lượng và độ tin cậy.

Về unit testing, các service functions được test riêng biệt. Test framework Jest được sử dụng cho cả Backend và Frontend. Backend tests cover user service với test cases tạo user mới thành công, validate email format, reject duplicate email, và hash password đúng cách. Product service được test với tìm kiếm products theo keywords, lọc theo category và price range, và pagination hoạt động đúng. Frontend component tests sử dụng React Testing Library, test Button component render đúng text và onClick được gọi, test ProductCard hiển thị product info và add to cart button hoạt động.

Về integration testing, API endpoints được test với Supertest. Auth endpoints test POST /auth/register tạo user và trả token, POST /auth/login với credentials đúng return token, login với wrong password return 401, và GET /auth/me với valid token return user info. Product endpoints test GET /products return array, GET /products với category filter return filtered results, POST /products without auth return 401, và POST /products với admin role tạo product. Order endpoints test POST /orders/cart thêm item thành công, GET /orders/cart return user's cart, và POST /orders/checkout chuyển cart thành pending order.

Về system testing, end-to-end user flows được test manually và với automation. User registration và login flow test user register với valid info, login với credentials, token được lưu vào cookie, và authenticated requests include token. Product browsing và search test user xem product list, filter theo category, search theo keyword, và pagination hoạt động. Shopping cart flow test user thêm product vào cart, cart persists sau refresh do localStorage, update quantity, remove items, và cart sync với backend. Checkout và payment flow test user checkout từ cart, nhập shipping info, redirect đến PayOS, hoàn tất payment trên PayOS, webhook cập nhật order status, và user redirect về success page.

Về payment testing, PayOS integration được test trong sandbox environment. Test cases bao gồm create payment link success với valid data, payment link redirect đúng URL, user complete payment trên PayOS test environment, webhook được gọi với correct payload, checksum verification pass, order status update từ pending sang paid, và payment failed scenario update status thành cancelled.

Về performance testing, load testing được thực hiện với concurrent requests. API response time được measure với simple GET requests < 200ms, complex queries với filters < 500ms, và POST requests với database writes < 1s. Database query performance được optimize với indexes, product search với text index nhanh hơn, và category filter sử dụng index hiệu quả. Frontend performance được evaluate với Lighthouse scores, First Contentful Paint < 1.5s, Time to Interactive < 3s, và Cumulative Layout Shift < 0.1.

Về security testing, các attack vectors được test. SQL injection không áp dụng do MongoDB, nhưng NoSQL injection được prevent bằng Mongoose validation. XSS được prevent bằng React escape output mặc định. CSRF được mitigate bằng JWT stateless authentication. Password security được verify với bcrypt hash với cost factor 10, passwords không bao giờ log hoặc return trong API, và comparePassword method hoạt động đúng. JWT security test token expiry enforcement, invalid tokens rejected, và token không chứa sensitive data.

Về kết quả đánh giá, chức năng đã implement đầy đủ theo requirements ở Chương 2. Authentication và authorization hoạt động đúng với role-based access control. Product management complete với CRUD operations, search và filter. Shopping cart với real-time sync giữa client và server. Payment integration với PayOS thành công với webhook handling. Admin dashboard hiển thị statistics chính xác.

Về limitations phát hiện, concurrent cart updates có thể gây race condition nếu user update từ nhiều tabs, cần implement optimistic locking. Payment webhook có thể miss nếu server down đúng lúc PayOS gọi, cần retry mechanism. Image upload không có resize hoặc compression, ảnh lớn impact performance. Search chỉ support tiếng Việt có dấu, không support tiếng Việt không dấu hoặc fuzzy search.

Tổng thể, hệ thống đạt được các mục tiêu đề ra, hoạt động stable trong testing, và sẵn sàng cho deployment production với minor improvements cần thiết.

\end{document}
