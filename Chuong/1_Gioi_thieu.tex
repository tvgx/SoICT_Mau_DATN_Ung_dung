\documentclass[../main.tex]{subfiles}
\begin{document}

Trong bối cảnh phát triển phần mềm hiện đại, việc triển khai ứng dụng web một cách nhanh chóng, tin cậy và nhất quán là yếu tố then chốt quyết định thành công của dự án. Tuy nhiên, quy trình triển khai thủ công truyền thống vẫn còn nhiều hạn chế về thời gian, tỷ lệ lỗi và khả năng mở rộng. Continuous Integration và Continuous Deployment (CI/CD) đã trở thành giải pháp quan trọng giúp tự động hóa quy trình này, nâng cao chất lượng và tốc độ phát triển phần mềm.

\section{Đặt vấn đề}
\label{section:1.1}

Trong quá trình phát triển ứng dụng web, việc triển khai từ môi trường phát triển lên môi trường production là một trong những giai đoạn quan trọng nhất. Tuy nhiên, quy trình triển khai thủ công truyền thống vẫn đang được sử dụng phổ biến và gặp phải nhiều vấn đề nghiêm trọng.

Về thời gian triển khai, quy trình thủ công thường mất từ 30 đến 45 phút cho mỗi lần deploy. Điều này bao gồm các bước: kéo code mới từ repository, cài đặt dependencies, build ứng dụng, cấu hình môi trường, khởi động lại server và kiểm tra hoạt động. Thời gian dài này không chỉ làm chậm quá trình phát triển mà còn hạn chế tần suất cập nhật, thường chỉ có thể deploy một lần mỗi tuần.

Về tỷ lệ lỗi, triển khai thủ công dễ xảy ra sai sót do phụ thuộc vào con người. Các lỗi phổ biến bao gồm: quên cài đặt dependencies mới, cấu hình sai biến môi trường, deploy nhầm branch, hoặc bỏ qua bước migration database. Những lỗi này thường chỉ được phát hiện sau khi deploy lên production, gây gián đoạn dịch vụ và ảnh hưởng đến người dùng.

Về khả năng rollback, khi phát hiện lỗi sau deploy, việc quay lại phiên bản cũ bằng tay mất từ một đến hai giờ. Trong thời gian này, hệ thống có thể bị gián đoạn hoặc hoạt động không ổn định, gây thiệt hại cho doanh nghiệp.

Về testing, do thời gian triển khai dài và phức tạp, nhiều developer có xu hướng bỏ qua các bước kiểm thử tự động để tiết kiệm thời gian. Điều này dẫn đến chất lượng code giảm và tăng nguy cơ lỗi trong production.

Về tính nhất quán, môi trường development, staging và production thường khác nhau về cấu hình, phiên bản dependencies, hoặc cách cài đặt. Sự khác biệt này gây ra tình trạng "works on my machine" - ứng dụng chạy tốt trên máy developer nhưng lỗi khi deploy lên server.

Từ những vấn đề trên, nhu cầu tự động hóa quy trình CI/CD trở nên cấp thiết. Một hệ thống CI/CD hiệu quả cần đáp ứng các yêu cầu: giảm thời gian build và deploy xuống còn vài phút, tự động chạy tests để đảm bảo chất lượng code, giảm tỷ lệ lỗi triển khai thông qua quy trình chuẩn hóa, hỗ trợ rollback nhanh chóng khi có sự cố, và đảm bảo tính nhất quán giữa các môi trường thông qua containerization.

\section{Mục tiêu đề tài}
\label{section:1.2}

Để giải quyết các vấn đề về triển khai thủ công nêu trên, đề tài hướng đến việc xây dựng một hệ thống CI/CD hoàn chỉnh với các mục tiêu cụ thể, có thể đo lường và kiểm chứng.

\subsection{Mục tiêu 1: Giảm thời gian build và deploy}

Giảm thời gian triển khai từ 30-45 phút (quy trình thủ công) xuống còn dưới 5 phút (quy trình tự động), đạt mức cải thiện tối thiểu 83\%. Thời gian này được tính từ khi developer push code lên repository cho đến khi ứng dụng được deploy thành công lên môi trường production và vượt qua health check. Mục tiêu được đo lườngqua metrics: thời gian trung bình của CI/CD pipeline execution, lead time từ commit đến deployment, và deployment frequency tăng từ weekly lên minimum 2-3 lần mỗi ngày.

\subsection{Mục tiêu 2: Tự động hóa kiểm thử}

Đạt được test coverage tối thiểu 80\% cho codebase, với tất cả tests được chạy tự động trong CI pipeline. Thời gian chạy toàn bộ test suite không vượt quá 3 phút để đảm bảo feedback nhanh chóng. Các loại tests bao gồm: unit tests cho business logic, integration tests cho API endpoints, và type checking với TypeScript. Mục tiêu được đo lường qua: code coverage percentage từ test reports, số lượng test cases passed/failed, và test execution time trong CI logs.

\subsection{Mục tiêu 3: Giảm tỷ lệ lỗi triển khai}

Đạt tỷ lệ deployment thành công tối thiểu 95\%, nghĩa là trong 100 lần deploy chỉ có tối đa 5 lần thất bại do lỗi technical. Các deployment failures được xác định qua health check endpoint, application logs, và user-reported issues trong vòng 24 giờ sau deploy. Hệ thống cần hỗ trợ rollback nhanh chóng trong vòng 5 phút nếu phát hiện lỗi. Mục tiêu được đo lường qua: deployment success rate, mean time to recovery (MTTR), và số lượng rollbacks cần thiết.

\subsection{Mục tiêu 4: Tối ưu hóa Docker image}

Giảm kích thước Docker image tối thiểu 70\% thông qua multi-stage builds và các kỹ thuật optimization. Cụ thể, từ kích thước ban đầu khoảng 1.2GB (với full Node.js dependencies) xuống còn dưới 350MB (chỉ runtime dependencies). Điều này giúp giảm thời gian pull image từ registry, tiết kiệm băng thông và storage. Build time với layer caching phải dưới 5 phút. Mục tiêu được đo lường qua: final image size từ Docker registry, build duration từ CI logs, và layer cache hit rate.

\subsection{Mục tiêu 5: Tăng deployment frequency}

Tăng deployment frequency từ weekly (1 lần/tuần do overhead của manual process) lên minimum 2-3 lần mỗi ngày. Điều này cho phép faster iteration, quicker bug fixes, và shorter feedback cycles. Hệ thống cần support continuous deployment cho staging environment (mỗi khi merge vào develop branch) và controlled deployment cho production (mỗi khi merge vào main branch). Mục tiêu được đo lường qua: số lượng successful deployments per day/week, deployment frequency trend theo thời gian.

\section{Phạm vi đề tài}
\label{section:1.3}

Phạm vi nghiên cứu của đề tài tập trung vào việc xây dựng và triển khai hệ thống CI/CD cho một ứng dụng web thương mại điện tử sử dụng kiến trúc Client-Server.

\subsection{Trong phạm vi đề tài}

Về CI/CD pipeline, đề tài triển khai đầy đủ workflow tự động với GitHub Actions làm nền tảng CI/CD chính. Pipeline bao gồm các giai đoạn: linting và type checking, unit testing và integration testing, Docker image build với multi-stage optimization, deployment tự động lên staging (develop branch) và production (main branch), health check sau deployment, và khả năng rollback.

Về containerization, hệ thống sử dụng Docker để đảm bảo tính nhất quán giữa các môi trường. Dockerfile được thiết kế theo multi-stage pattern để tối ưu kích thước image. Base image sử dụng node:18-alpine để giảm footprint. Non-root user được cấu hình để tăng cường bảo mật.

Về deployment platforms, backend được deploy lên Render với auto-deployment và health check integration. Frontend được deploy lên Vercel với preview deployments cho pull requests và edge CDN distribution.

Về ứng dụng web, hệ thống bao gồm frontend với Next.js (App Router, Server-Side Rendering), backend với Express.js (RESTful API, MVC architecture), cơ sở dữ liệu MongoDB Atlas, và tích hợp PayOS cho thanh toán.

Về test environments, đề tài triển khai ba môi trường: development (local), staging (develop branch, sandbox services), và production (main branch, production services).

\subsection{Ngoài phạm vi đề tài}

Một số công nghệ và tính năng không nằm trong phạm vi nghiên cứu, bao gồm: Kubernetes orchestration cho container management (được đề cập như hướng phát triển), blue-green deployment strategy (hiện tại sử dụng rolling deployment), comprehensive monitoring tools như Prometheus/Grafana (chỉ có basic logging), infrastructure as code với Terraform/Ansible, ứng dụng mobile native (iOS/Android), và multiple cloud providers (chỉ tập trung Render và Vercel).

\section{Cấu trúc báo cáo}
\section{Cấu trúc báo cáo}
\label{section:1.4}

Phần còn lại của báo cáo được tổ chức thành năm chương tiếp theo, mỗi chương tập trung vào một khía cạnh cụ thể của đềài liên quan đến CI/CD và phát triển ứng dụng web.

Chương 2 trình bày quá trình khảo sát và phân tích yêu cầu. Chương này bắt đầu với hiện trạng triển khai thủ công, so sánh với quy trình CI/CD tự động qua bảng metrics cụ thể. Tiếp theo là phân tích các yêu cầu chức năng của hệ thống CI/CD như build, test, deploy tự động. Cuối cùng đề cập các yêu cầu phi chức năng về thời gian pipeline, success rate, và image size.

Chương 3 giới thiệu các công nghệ và kiến trúc được áp dụng. Phần đầu trình bày kiến trúc CI/CD tổng thể với luồng từ code commit đến deployment. Tiếp theo giới thiệu các công nghệ chính như GitHub Actions, Docker, Render và Vercel, kèm theo lý do lựa chọn và trade-offs so với các giải pháp khác. Phần cuối mô tả ngắn gọn về kiến trúc ứng dụng Frontend và Backend.

Chương 4 trình bày triển khai và đánh giá hệ thống. Phần  triển khai mô tả chi tiết CI/CD pipeline implementation, Docker containerization, và deployment automation. Phần đánh giá là trọng tâm của chương, bao gồm bảng đánh giá kết quả so với mục tiêu đề ra, metrics cụ thể về deployment time, success rate, image size, và testing results.

Chương 5 kết luận toàn bộ đề tài. Nội dung bao gồm tổng kết các đóng góp chính tập trung vào CI/CD (70\%) và application development (30\%). Phần hạn chế trình bày trung thực những gì chưa hoàn thiện như auto-rollback, monitoring tools, và test coverage. Bài học kinh nghiệm chia sẻ những khó khăn thực tế về pipeline optimization, Docker caching, và testing strategy. Cuối cùng là đề xuất hướng phát triển như Kubernetes, blue-green deployment, và comprehensive monitoring.

\end{document}